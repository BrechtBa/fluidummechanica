% Dit werk is gelicenseerd onder de licentie Creative Commons Naamsvermelding-GelijkDelen 4.0 Internationaal. Ga naar http://creativecommons.org/licenses/by-sa/4.0/ om een kopie van de licentie te kunnen lezen.
% vaak gebruikte packages, nederlands
\usepackage[margin=2.5cm]{geometry}     % Marges instellen
\usepackage[dutch]{babel}               % Voor nederlandstalige hyphenatie (woordsplitsing)
\usepackage[utf8]{inputenc}             % Om niet ascii karakters rechtstreeks te kunnen typen
\usepackage{fancyhdr}                   % Voor fancy headers en footers
\usepackage{parskip}                    % Om paragrafen met een verticale spatie te laten beginnen
\usepackage[nottoc]{tocbibind}			% Bibliografie en inhoudsopgave in ToC
\usepackage{amsmath,amsthm}             % Uitgebreide wiskundige mogelijkheden
\usepackage{graphicx,subfigure}         % Om figuren te kunnen verwerken
\usepackage[small,bf,hang]{caption}     % Om de captions wat te verbeteren
\usepackage[section]{placeins}			% Om ervoor te zorgen dat floats binnen dezelfde section blijven
\usepackage{color}						% Om kleuren te gebruiken
\usepackage{framed}						% Om tekst te omkaderen
\usepackage{multicol}					% Om te switchen tussen enkele of meerdere kolommen
\usepackage{url}                        % Om url's te verwerken
\usepackage[pdftex,                     % Om hyperlinks en metadata te hebben in het pdfdocument.
			plainpages=false,
            pdfauthor={Brecht Baeten},
            pdftitle={Fluidummechanica}]{hyperref}


%%%%%%%%%%%%%%%%%%%%%%%%%%%%%%%%%%%%%%%%%%%%%%%%%%%%%%%%%%%%
% Algemene instellingen van het document.
%%%%%%%%%%%%%%%%%%%%%%%%%%%%%%%%%%%%%%%%%%%%%%%%%%%%%%%%%%%%
\renewcommand{\baselinestretch}{1.2} 	% De interlinie afstand wat vergroten.
\setcounter{MaxMatrixCols}{50}          % Max 20 kolommen in een matrix


%%%%%%%%%%%%%%%%%%%%%%%%%%%%%%%%%%%%%%%%%%%%%%%%%%%%%%%%%%%%
% Headers en footers
%%%%%%%%%%%%%%%%%%%%%%%%%%%%%%%%%%%%%%%%%%%%%%%%%%%%%%%%%%%%
\pagestyle{fancy}
\fancyhf{}
\renewcommand{\headrulewidth}{0pt}
\fancyhead[RO] {\rightmark}
\fancyhead[LE] {\leftmark}
\fancyfoot[RO,LE] {\thepage}

% no dot after chapter number
\renewcommand{\chaptermark}[1]{
	\markboth{\MakeUppercase{ \chaptername\ \thechapter\quad #1}}{}
}
% no dot after section number
\renewcommand{\sectionmark}[1]{
	\markright{\MakeUppercase{ \thesection\quad #1}}{}
}

% page header and footer style in mainmatter aanpassen
\let\newmainmatter\mainmatter
\renewcommand{\mainmatter}{

	\pagestyle{fancy}
	\fancyhf{}
	\renewcommand{\headrulewidth}{0pt}
	\fancyhead[RO] {\rightmark}
	\fancyhead[LE] {\leftmark}
	\fancyfoot[RO,LE] {\thepage}
	\fancyfoot[C]{\includegraphics[height=0.15cm]{fig/cc}
				  \includegraphics[height=0.15cm]{fig/by}
				  \includegraphics[height=0.15cm]{fig/sa}
				  \quad \tiny{Brecht Baeten}}

	\newmainmatter
}
\let\newappendix\appendix
\renewcommand{\appendix}{
	\fancyfoot{}
	\fancyfoot[RO,LE] {\thepage}
	\newappendix
}


%%%%%%%%%%%%%%%%%%%%%%%%%%%%%%%%%%%%%%%%%%%%%%%%%%%%%%%%%%%%
% Nieuwe omgevingen
%%%%%%%%%%%%%%%%%%%%%%%%%%%%%%%%%%%%%%%%%%%%%%%%%%%%%%%%%%%%
% Voorbeeld
\definecolor{shadecolor}{gray}{0.95}
\newcounter{voorbeeldcounter}[chapter]
\renewcommand{\thevoorbeeldcounter}{\thechapter.\arabic{voorbeeldcounter}}
\makeatletter
\newenvironment{voorbeeld}
{
\vspace{3mm}
\addtolength{\leftskip}{5mm}
\begin{shaded*}
\vspace{-3mm}
\refstepcounter{voorbeeldcounter}
\noindent
\textbf{Voorbeeld \thevoorbeeldcounter:\\}
%\vspace{-8mm}
%\begin{multicols}{2}
}
{
%\end{multicols}
\end{shaded*}
\addtolength{\leftskip}{-5mm}
}
\makeatother  