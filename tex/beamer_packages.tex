%%%%%%%%%%%%%%%%%%%%%%%%%%%%%%
% Packages
%%%%%%%%%%%%%%%%%%%%%%%%%%%%%%

\usepackage{geometry}              		 % 
\usepackage[dutch]{babel}                % Voor nederlandstalige hyphenatie (woordsplitsing)
\uselanguage{dutch}
\languagepath{dutch}
\usepackage{amsmath,amsthm}              % Uitgebreide wiskundige mogelijkheden
\usepackage{url}                         % Om url's te verwerken
\usepackage{graphicx,subfigure}          % Om figuren te kunnen verwerken
\usepackage{color}						 % Om kleuren in Inkscape figuren te kunnen weergeven
\usepackage[utf8]{inputenc}              % Om niet ascii karakters rechtstreeks te kunnen typen
\usepackage{float}                       % Om nieuwe float environments aan te maken. Ook optie H!
\usepackage[section]{placeins}			 % Om ervoor te zorgen dat floats binnen dezelfde section blijven
\usepackage{eurosym}                     % om het euro symbool te krijgen
\usepackage{textcomp}                    % Voor onder andere graden celsius
%\usepackage{fancyhdr}                    % Voor fancy headers en footers
\usepackage{parskip}                     % Om paragrafen met een verticale spatie ipv horizontaal te laten beginnen
\usepackage{multicol}
%\usepackage[plainpages=false]{hyperref}  % Om hyperlinks te hebben in het pdfdocument
\usepackage[absolute,overlay]{textpos}

%%%%%%%%%%%%%%%%%%%%%%%%%%%%%%
% Layout
%%%%%%%%%%%%%%%%%%%%%%%%%%%%%%
\usetheme{Frankfurt}
\AtBeginSection[]
{
  \begin{frame}
    \frametitle{Inhoud}
    \tableofcontents[currentsection]
  \end{frame}
}

%%%%%%%%%%%%%%%%%%%%%%%%%%%%%%
% Omgevingen
%%%%%%%%%%%%%%%%%%%%%%%%%%%%%%
