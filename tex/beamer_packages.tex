%%%%%%%%%%%%%%%%%%%%%%%%%%%%%%
% Packages
%%%%%%%%%%%%%%%%%%%%%%%%%%%%%%

\usepackage{geometry}              		 % 
\usepackage[dutch]{babel}                % Voor nederlandstalige hyphenatie (woordsplitsing)
\uselanguage{dutch}
\languagepath{dutch}
\usepackage{amsmath,amsthm}              % Uitgebreide wiskundige mogelijkheden
\usepackage{url}                         % Om url's te verwerken
\usepackage{graphicx,subfigure}          % Om figuren te kunnen verwerken
\usepackage{color}						 % Om kleuren in Inkscape figuren te kunnen weergeven
\usepackage[utf8]{inputenc}              % Om niet ascii karakters rechtstreeks te kunnen typen
\usepackage{float}                       % Om nieuwe float environments aan te maken. Ook optie H!
\usepackage[section]{placeins}			 % Om ervoor te zorgen dat floats binnen dezelfde section blijven
\usepackage{eurosym}                     % om het euro symbool te krijgen
\usepackage{textcomp}                    % Voor onder andere graden celsius
%\usepackage{fancyhdr}                    % Voor fancy headers en footers
\usepackage{parskip}                     % Om paragrafen met een verticale spatie ipv horizontaal te laten beginnen
\usepackage{multicol}
%\usepackage[plainpages=false]{hyperref}  % Om hyperlinks te hebben in het pdfdocument
\usepackage[absolute,overlay]{textpos}

%%%%%%%%%%%%%%%%%%%%%%%%%%%%%%
% Layout
%%%%%%%%%%%%%%%%%%%%%%%%%%%%%%
\usetheme{Frankfurt}
\AtBeginSection[]
{
  \begin{frame}
    \frametitle{Inhoud}
    \tableofcontents[currentsection]
  \end{frame}
}

%%%%%%%%%%%%%%%%%%%%%%%%%%%%%%
% Omgevingen
%%%%%%%%%%%%%%%%%%%%%%%%%%%%%%


%%%%%%%%%%%%%%%%%%%%%%%%%%%%%%
% Nieuwe commandos
%%%%%%%%%%%%%%%%%%%%%%%%%%%%%%

% De differentiaal operator
\newcommand{\diff}{\ensuremath{\mathrm{d}}}
\newcommand{\subsdiff}{\ensuremath{\mathrm{D}}}
\newcommand{\vardiff}{\ensuremath{\mathrm{\delta}}}

% Super en subscript
\newcommand{\supsc}[1]{\ensuremath{^{\text{#1}}}}   % Superscript in tekst
\newcommand{\subsc}[1]{\ensuremath{_{\text{#1}}}}   % Subscript in tekst

% Vectoren en matrices
\newcommand{\vt}[1]{\ensuremath{\boldsymbol{#1}}} % vector in juiste lettertype
\newcommand{\mx}[1]{\ensuremath{\mathsf{#1}}}	  % matrix in juiste lettertype

% Nieuw commando om iets te benadrukken en tegelijkertijd in de index te steken.
\newcommand{\begrip}[1]{\index{#1}\textbf{#1}\xspace}

% Graden celcius
\newcommand{\degC}{\ensuremath{^\circ \mathrm{C}}}


% nieuw commando om svg files dynamisch te updaten
\newcommand{\executeiffilenewer}[3]{%
\ifnum\pdfstrcmp{\pdffilemoddate{#1}}%
{\pdffilemoddate{#2}}>0%
{\immediate\write18{#3}}\fi%
}
% nieuw commando om. svg figuren in te voegen
% Gebruik: \includesvg{path/filename.svg}
\newcommand{\includesvg}[2][0]{%
\executeiffilenewer{#2.svg}{#2.pdf}%
{inkscape -z -C --file=#2.svg %
--export-pdf=#2.pdf --export-latex}%
\ifx#10
	\let\svgwidth\undefined
\else
	\def\svgwidth{#1}
\fi%
\input{#2.pdf_tex}%
\ifx \svgwidth\undefined
\else
	\let\svgwidth\undefined
\fi%
}

% nieuw commando om .fig figuren in te voegen
\newcommand{\includefig}[2][0]{%
\ifx#10
	\let\figwidth\undefined
\else
	\def\figwidth{#1}
\fi%
\input{#2.pdf_tex}%
\ifx \figwidth\undefined
\else
	\let\figwidth\undefined
\fi%
}
