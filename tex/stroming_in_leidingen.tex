\FloatBarrier
\chapter{Stroming in leidingen}
\label{sec:Stroming in leidingen}
\begin{toepassing}
	\label{buis}
	Water van 20\textcelsius\ vloeit in een buis van \unit{40}{mm} diameter over een afstand van \unit{500}{m}. Het debiet is \unit{3}{liter/s}.  De ruwheid van de buis is \unit{0.046}{mm}.
		
	Bepaal de drukval over de buis voor turbulente stroming.
		
	Wat zou de drukval zijn indien we er zouden in slagen om de stroming laminair te houden?
\end{toepassing}
\begin{antwoord}
	$\Delta p = \unit{784}{kPa}$, $\Delta p_{\text{laminair}} = \unit{23}{kPa}$
\end{antwoord}
%%%%%%%%%%%%%%%%%%%%%%%%%%%%%%%%%%%%%%%%%%%%%%%%%%%%%%%%%%%%%%%%%%
\begin{toepassing}
	\label{buisdiameter}
	Een buis met ruwheid $k = \unit{0.0015}{mm}$ moet \unit{2}{liter/s} water van 20\textcelsius\ transporteren over een afstand van \unit{400}{m}.  Het verlies in drukhoogte mag hoogstens \unit{30}{m} bedragen.
		
	Bepaal de minimale diameter van de buis.
\end{toepassing}
\begin{antwoord}
	$d = \unit{39}{mm}$
\end{antwoord}
%%%%%%%%%%%%%%%%%%%%%%%%%%%%%%%%%%%%%%%%%%%%%%%%%%%%%%%%%%%%%%%%%%
\begin{toepassing}
	\label{hevel met verlies}
	Een hevel wordt gebruikt om water uit een tank te halen, aan het uiteinde van de hevel heerst de atmosfeerdruk. De hoogtes zijn zoals aangegeven op onderstaande figuur, de buis heeft een constante diameter. De gebruikte buis heeft een diameter van \unit{24}{mm} een ruwheid van \unit{0.1}{mm} en een lengte van \unit{3}{m}, alle bochten hebben een straal van \unit{40}{mm}, de instroming mag verliesvrij verondersteld worden. De viscositeit van water is \unit{1}{mm^2/s}.
		
	Bepaal het debiet.
	\begin{center}
		\includesvg{fig/stroming_in_leidingen/hevel}
	\end{center}
\end{toepassing}
\begin{antwoord}
	$\dot{V} = \unit{44.5}{l/min}$
\end{antwoord}
%%%%%%%%%%%%%%%%%%%%%%%%%%%%%%%%%%%%%%%%%%%%%%%%%%%%%%%%%%%%%%%%%%
\begin{toepassing}
	\label{ventilatiekanaal}
	Voor een ventilatiekanaal van in een vloer in een woning, waardoor een luchtdebiet van \unit{80}{m^3/h} stroomt, heeft men de keuze tussen een rechthoekig kanaal met hoogte 60mm en breedte 200mm en ruwheid 0.1mm of een aantal parallelle buizen met diameter 60mm en ruwheid 0.02mm. 
		
	Bepaal het aantal buizen dat nodig is om een lagere drukval te genereren. $\rho = \unit{1.22}{kg/m^3}$, $\nu = \unit{15}{mm^2/s}$, verwaarloos verliezen aan in- en uitstroming.
\end{toepassing}
\begin{antwoord}
	$n = 3$
\end{antwoord}
%%%%%%%%%%%%%%%%%%%%%%%%%%%%%%%%%%%%%%%%%%%%%%%%%%%%%%%%%%%%%%%%%%
\begin{toepassing}
	\label{pompopvoerhoogte}
	Water wordt door leidingen zoals op de figuur aangegeven van een lager naar een hoger gelegen reservoir gepompt. Alle leidingen hebben een ruwheid van \unit{0.5}{mm}. De pomp heeft een karakteristiek zoals aangegeven op de figuur. De bocht heeft een gemiddelde straal van \unit{120}{mm}, de intrede heeft een verliescoëfficiënt $\xi = 0.1$, de uittrede heeft een verliesco\"effici\"ent $\xi = 0.8$.
		
	Bepaal het debiet en het werkingspunt van de pomp.
		
	Teken het verloop van de hoogte, snelheidshoogte, drukhoogte en het drukhoogte verlies op \'e\'en figuur.
	\begin{center}
		\includesvg{fig/stroming_in_leidingen/pompopvoerhoogte}
	\end{center}
\end{toepassing}
\begin{antwoord}
	$\dot{V} = \unit{155}{m^3/h}$
\end{antwoord}
%%%%%%%%%%%%%%%%%%%%%%%%%%%%%%%%%%%%%%%%%%%%%%%%%%%%%%%%%%%%%%%%%%%%%%%%%%%%%%%%%%%%
\begin{toepassing}
	\label{olieafvoerdimensionering}
	In een machine dient olie met een dichtheid van \unit{830}{kg/m^3} en viscositeit van \unit{240}{mm^2/s} afgevoerd te worden uit een carter. Door de smering van verschillende componenten stroomt er een debiet van \unit{80}{l/min} naar het carter. De afvoerleiding dient een pad te volgen zoals aangegeven op de tekening. Bochten in de leiding worden uitgevoerd met een straal gelijk aan de buisdiameter. De druk in de machine en aan de olieafvoer is atmosfeerdruk.
	
	Bepaal de nodige leiding diameter zodat het niveau in het carter niet hoger komt dan 500mm boven de afvoer.
	\begin{center}
		\includesvg{fig/stroming_in_leidingen/olieafvoerdimensionering}
	\end{center}
\end{toepassing}
\begin{antwoord}
	$\dot{d} = \unit{42}{mm}$
\end{antwoord}
%%%%%%%%%%%%%%%%%%%%%%%%%%%%%%%%%%%%%%%%%%%%%%%%%%%%%%%%%%%%%%%%%%%%%%%%%%%%%%%%%%%%
	\section*{Antwoorden}
\begin{multicols}{2}
	\includecollection{antwoorden}
\end{multicols}