\chapter{Studeerplan}
Hier wordt aangegeven welke toepassingen in de les zullen behandeld worden en voor welke toepassingen het aangeraden is om ze thuis zelf te maken.
	\section*{Sessie 1}
Voor de start van sessie 1 dient men de theorie van Hoofdstuk 1 en Hoofdstuk 2 te beheersen.

Tijdens de BKV sessie zullen toepassingen \ref{u-buis_met_hoogteverschil}, \ref{stormvloedkering} en \ref{sluisklep} behandeld worden.

Het is aanbevolen om toepassingen \ref{rotatiegieten}, \ref{wrijvingskracht} en \ref{betonnenligger} zeker te maken.

Toepassing \ref{boei} en \ref{containerschip} zijn van een hogere moeilijkeidsgraad en kunnen als extra oefeningen aanzien worden.
	
	\section*{Sessie 2}
Voor de start van sessie 2 dient men de theorie van Hoofdstuk 3 en Hoofdstuk 4 te beheersen.

Tijdens de BKV sessie zullen toepassingen \ref{brandslang}, \ref{hevel} en \ref{45gradenbocht} behandeld worden.

Het is aanbevolen om toepassingen \ref{waterstraal}, \ref{waterkraan}, \ref{turbine} en \ref{diffusiebocht} zeker te maken.

	\section*{Sessie 3}
Voor de start van sessie 3 dient men de theorie van Hoofdstuk 5 te beheersen en de theorie van Hoofdstuk 6 doorgenomen te hebben.

Tijdens de BKV sessie zullen toepassingen \ref{weerballon}, \ref{onderzeeer} en \ref{airbus} behandeld worden.

	\section*{Sessie 4}
Voor de start van sessie 4 dient men de theorie van Hoofdstuk 8 te beheersen.

Tijdens de BKV sessie zullen toepassingen \ref{buisdiameter}, \ref{hevel met verlies} en \ref{pompopvoerhoogte} behandeld worden.
	
	\section*{Sessie 5}
Voor de start van sessie 5 dient men de theorie van Hoofdstuk 9 te beheersen.

Tijdens de BKV sessie zullen toepassingen \ref{3_reservoirs}, \ref{geperforeerde leiding} en \ref{grondwarmtewisselaar} behandeld worden.