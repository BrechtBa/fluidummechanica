\chapter{Studeerplan}
Hier wordt aangegeven welke toepassingen in de les zullen behandeld worden en welke voorkennis voor elke oefeningensessie vereist is.

De toepassingen met een "*" worden aangeraden om zeker zelf te maken.
	\section*{Sessie \ref{sec:Hydrostatica}}
Voor de start van sessie \ref{sec:Hydrostatica} dient men de theorie van Hoofdstuk 1 en Hoofdstuk 2 te beheersen.

Leerdoelstellingen:
\begin{itemize}
	\item De student kan de hydrostatische wet toepassen om drukverschillen door hoogteverschillen in verschillende media te bepalen
	\item De student kan krachten en momenten op vlakke oppervlakken ten gevolge van hydrostatische druk uitrekenen
	\item De student kan krachten op eenvoudige gebogen oppervlakken ten gevolge van hydrostatische druk uitrekenen
\end{itemize}
	
	\section*{Sessie \ref{sec:Behoudsvergelijkingen}}
Voor de start van sessie \ref{sec:Behoudsvergelijkingen} dient men de theorie van Hoofdstuk 3 te beheersen.

Leerdoelstellingen:
\begin{itemize}
	\item De student kan een geschikt controlevolume definiëren om een analyse uit te voeren
	\item De student kan debieten of snelheden bepalen met behulp van de continuïteits vergelijking
	\item De student kan krachten ten gevolgen van stromingen en drukken uitrekenen aan de hand van het snelheidsveld
\end{itemize}

	\section*{Sessie \ref{sec:Behoudsvergelijkingen gecombineerd}}
Voor de start van sessie \ref{sec:Behoudsvergelijkingen gecombineerd} dient men de theorie van Hoofdstuk 3 en Hoofdstuk 4 te beheersen.

Leerdoelstellingen:
\begin{itemize}
	\item De student weet wanneer de vergelijking van Bernoulli toegepast mag worden en kan deze toepassen
	\item De student kan de behoudsvergelijkingen combineren om een systeem te analyseren
\end{itemize}

	\section*{Sessie \ref{sec:Gelijkvormigheid en uitwendige stroming}}
Voor de start van sessie \ref{sec:Gelijkvormigheid en uitwendige stroming} dient men de theorie van Hoofdstuk 5 te beheersen en de theorie van Hoofdstuk 6 doorgenomen te hebben.

Leerdoelstellingen:
\begin{itemize}
	\item De student kan dimensieloze getallen gebruiken om een gelijkvormig experiment op schaal te definieren
	\item De student kan dimensieloze getallen gebruiken om resultaten van experimenten te vertalen naar gelijkvormige situaties
	\item De student kan weerstands en liftcoëfficiënten gebruiken om krachten te berekenen
\end{itemize}

	\section*{Sessie \ref{sec:Stroming in leidingen}}
Voor de start van sessie \ref{sec:Stroming in leidingen} dient men de theorie van Hoofdstuk 8 te beheersen.

Leerdoelstellingen:
\begin{itemize}
	\item De student kan bepalen of de stroming in een leiding laminair of turbulent is
	\item De student kan drukverliezen in een leiding ten gevolge van stroming van een viskeuze vloeistof bepalen
	\item De student kan een leiding dimensioneren op basis van een maximaal drukverlies
	\item De student kan het debiet in een leiding gedreven door een pomp bepalen
\end{itemize}

	\section*{Sessie \ref{sec:Leidingnetwerken}}
Voor de start van sessie \ref{sec:Leidingnetwerken} dient men de theorie van Hoofdstuk 9 te beheersen.

Leerdoelstellingen:
\begin{itemize}
	\item De student kan de drukval doorheen een combinatie van serie en parallelschakelingen van leidingen bepalen bij een gekend totaal debiet
	\item De student kan het debiet bepalen doorheen een combinatie van serie en parallelschakelingen van leidingen aangedreven door een pomp
	\item De student kan het debiet in een leidingnetwerk met constante inlaat drukken bepalen
\end{itemize}
