\chapter{Studeerplan}
Hier wordt aangegeven welke toepassingen in de les zullen behandeld worden en welke voorkennis voor elke oefeningensessie vereist is.

De toepassingen met een "*" worden aangeraden om zeker zelf te maken.
	\section*{Sessie 1}
Voor de start van sessie 1 dient men de theorie van Hoofdstuk 1 en Hoofdstuk 2 te beheersen.

Leerdoelstellingen:
\begin{itemize}
	\item De student kan de hydrostatische wet toepassen om drukverschillen door hoogteverschillen in verschillende media te bepalen
	\item De student kan krachten en momenten op vlakke oppervlakken ten gevolge van hydrostatische druk uitrekenen
	\item De student kan krachten op eenvoudige gebogen oppervlakken ten gevolge van hydrostatische druk uitrekenen
\end{itemize}
	
	\section*{Sessie 2}
Voor de start van sessie 2 dient men de theorie van Hoofdstuk 3 en Hoofdstuk 4 te beheersen.

	\section*{Sessie 3}
Voor de start van sessie 3 dient men de theorie van Hoofdstuk 5 te beheersen en de theorie van Hoofdstuk 6 doorgenomen te hebben.

	\section*{Sessie 4}
Voor de start van sessie 4 dient men de theorie van Hoofdstuk 8 te beheersen.
	
	\section*{Sessie 5}
Voor de start van sessie 5 dient men de theorie van Hoofdstuk 9 te beheersen.