\chapter{Formularium}
	\section{Hydrostatica}
\begin{multicols}{2}
		Hydrostatische druk:
		\begin{equation}
			\diff p = \rho g \diff z
		\end{equation}
		Hydrostatische kracht:
		\begin{equation}
			F = \int p \diff A
		\end{equation}
\end{multicols}

	\section{Behoudsvergelijkingen}
Behoudvergelijkingen in integraal vorm
\begin{equation}
	0 = \frac{\partial}{\partial t} \int_{CV} \rho \diff V + \int_{\partial CV} \rho (\vt{v}\cdot \vt{n}) \diff A
\end{equation}
\begin{equation}
	\sum \vt{F} = \frac{\partial}{\partial t} \int_{CV} \rho \vt{v} \diff V + \int_{\partial CV} \rho \vt{v} (\vt{v}\cdot \vt{n}) \diff A
\end{equation}
\begin{equation}
	\dot{Q}-\dot{W}_t = \frac{\partial}{\partial t} \int_{CV} \rho \left(u+\frac{v^2}{2}+g z \right) \diff V + \int_{\partial CV} \rho \left(u+\frac{p}{\rho}+\frac{v^2}{2}+g z \right) (\vt{v}\cdot \vt{n}) \diff A
\end{equation}
Behoudvergelijkingen bij een stationaire stroming met één instroming en één uitstroming:
\begin{equation}
	0 = \dot{m}_{uit}-\dot{m}_{in}
\end{equation}
\begin{equation}
	\sum \vt{F} = \dot{m} \left( \vt{v}_{uit}-\vt{v}_{in} \right)
\end{equation}
\begin{equation}
	\dot{Q}-\dot{W}_t = \dot{m} \left(u_{uit}-u_{in}+\frac{p_{uit}}{\rho_{uit}}-\frac{p_{in}}{\rho_{in}}+\frac{v_{uit}^2}{2}-\frac{v_{in}^2}{2}+g z_{uit}-g z_{in} \right)
\end{equation}
Behoudvergelijkingen in differentiaal vorm:
\begin{equation}
	\frac{\partial \rho}{\partial t} + \vt{\nabla} \cdot \vt {v} = 0
\end{equation}
\begin{equation}
	\rho \frac{\partial \vt{v}}{\partial t} + \rho \vt{v} \vt{\nabla} \cdot \vt {v} = -\vt{\nabla} p + \rho \vt{g} + \mu \vt{\nabla}^2 \vt{v}
\end{equation}
\begin{equation}
	\rho v \frac{\diff v}{\diff s} + \frac{\diff p}{\diff s} + \rho g \frac{\diff z}{\diff s} = 0
\end{equation}
Uitgebreide vergelijking van Bernoulli
\begin{equation}
	\frac{1}{2} \frac{v_1^2}{g} + \frac{p_1}{\rho g} + z_1 - h_L + h_P = \frac{1}{2} \frac{v_2^2}{g} + \frac{p_2}{\rho g} + z_2
\end{equation}

	\section{Dimensieloze getallen}
\begin{multicols}{2}
	\begin{equation}
		\text{Re} = \frac{\rho v D}{\mu} = \frac{v D}{\nu}
	\end{equation}
	\begin{equation}
		\text{Fr} = \frac{v}{\sqrt{g D}}
	\end{equation}
	\begin{equation}
		\text{Eu} = \frac{\Delta p}{\rho v^2}
	\end{equation}
	\begin{equation}
		\text{Ma} = \frac{v}{c}
	\end{equation}
	\begin{equation}
		C_p = \frac{p}{\frac{1}{2}\rho v^2} \cong \frac{p}{\rho N^2 D^2}
	\end{equation}
	\begin{equation}
		C_F = \frac{F}{\frac{1}{2}\rho v^2 A}
	\end{equation}
	\begin{equation}
		C_P = \frac{P}{\frac{1}{2}\rho v^3 D^2} \cong \frac{p}{\rho N^3 D^5}
	\end{equation}
	\begin{equation}
		C_p = \frac{\dot{V}}{v D^2} \cong \frac{\dot{V}}{N D^3}
	\end{equation}
\end{multicols}
	\section{Open kanaal stroming}
\begin{multicols}{2}
	Golfsnelheid:
	\begin{equation}
		c = \sqrt{gy}
	\end{equation}
	\begin{equation}
		c = \sqrt{\frac{g\lambda}{2\pi}}
	\end{equation}
	Kritische diepte:
	\begin{equation}
		y_c = \left(\frac{q^2}{g}\right)^{1/3}
	\end{equation}
\end{multicols}
	\section{Stroming in leidingen}
\begin{multicols}{2}
	Ladingsverlies bij stroming in een rechte cilindrische leiding met diameter D:
	\begin{equation}
		h_L = f \frac{v^2}{2 g}\frac{L}{D} = 8 f \frac{\dot{V}^2}{g \pi^2}\frac{L}{D^5} = R \dot{V}^2
	\end{equation}
	Wrijvingsfactor bij laminaire stroming in cilindrische buizen:
	\begin{equation}
		f = \frac{64}{\text{Re}}
	\end{equation}
	Wrijvingsfactor bij turbulente stroming in cilindrische buizen volgens Colebrook:
	\begin{equation}
		\frac{1}{\sqrt{f}} = -2 \log \left( \frac{\frac{\varepsilon}{D}}{3.71}+\frac{2.51}{\text{Re}\sqrt{f}} \right)
	\end{equation}
	Ladingsverlies bij stroming door leidingsonderdelen:
	\begin{equation}
		h_L = \zeta \frac{v^2}{2 g} = \zeta \frac{\dot{V}^2}{2 g A^2} = R \dot{V}^2
	\end{equation}
	Ladingsverlies bij een plotse verwijding:
	\begin{equation}
		h_L = \left(\frac{A_1}{A_2}-1\right)^2 \frac{v_1^2}{2 g} = \zeta \frac{v_1^2}{2 g}
	\end{equation}
	Gebruik van de correctiefactor op de verliescoëfficiënt:
	\begin{equation}
		\zeta_{corr} = k \zeta
	\end{equation}
\end{multicols}