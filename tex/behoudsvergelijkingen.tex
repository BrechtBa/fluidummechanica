\chapter{Behoudsvergelijkingen}
\label{sec:Behoudsvergelijkingen}
	\begin{toepassing}
		\label{buisaftap}
In een buis met een aftap met afmetingen zoals aangegeven op de figuur stroomt water.
		
Bepaal de gemiddelde stromingssnelheid in de aftap.
		\begin{center}
			\includesvg{fig/behoudsvergelijkingen/buisaftap}
		\end{center}
	\end{toepassing}
	\begin{antwoord}
		$v = \unit{1.3}{m/s}$
	\end{antwoord}
%%%%%%%%%%%%%%%%%%%%%%%%%%%%%%%%%%%%%%%%%%%%%%%%%%%%%%%%%%%%%%%%%%%%%%%%%%%%%%%%%%%%%%%%%
	\begin{toepassing}
		\label{afvoerbocht}
Een PVC bochtstuk van \unit{90}{\degree} voor afvoerleiding heeft een binnen diameter van \unit{32}{mm}. De hartlijn heeft een straal van \unit{20}{mm}. Door de buis stroomt water met een debiet van \unit{40}{l/min}.
		
Als er geen drukverliezen optreden in de bocht en de druk aan de uitgang is atmosfeerdruk, bepaal dan de grootte en richting van de kracht die door de stroming uitgeoefend wordt op de buis.
		\begin{center}
			\includesvg{fig/behoudsvergelijkingen/afvoerbocht}
		\end{center}
	\end{toepassing}
	\begin{antwoord}
		$F_x = \unit{0.55}{N}$, $F_y = \unit{0.55}{N}$
	\end{antwoord}
%%%%%%%%%%%%%%%%%%%%%%%%%%%%%%%%%%%%%%%%%%%%%%%%%%%%%%%%%%%%%%%%%%%%%%%%%%%%%%%%%%%%%%%%%
	\begin{toepassing*}
		\label{waterstraal}
Bereken de kracht die een vrije stationaire waterstraal die loodrecht op een wand spuit op de wand uitoefent.
		\begin{center}
			\includesvg{fig/behoudsvergelijkingen/waterstraal}
		\end{center}
	\end{toepassing*}
	\begin{antwoord}
		$F = \unit{90}{N}$
	\end{antwoord}
%%%%%%%%%%%%%%%%%%%%%%%%%%%%%%%%%%%%%%%%%%%%%%%%%%%%%%%%%%%%%%%%%%%%%%%%%%%%%%%%%%%%%%%%%
	\begin{toepassing*}
		\label{hevel}
Een hevel wordt gebruikt om water uit een tank te halen, aan het uiteinde van de hevel heerst de atmosfeerdruk. De hoogtes zijn zoals aangegeven op onderstaande figuur, de buis heeft een constante diameter. Veronderstel dat het water zich niet viskeus gedraagt.
		
Bepaal de gemiddelde snelheid en de druk op het hoogste punt in de hevel.
		\begin{center}
			\includesvg{fig/behoudsvergelijkingen/hevel}
		\end{center}
	\end{toepassing*}
	\begin{antwoord}
		$v = \unit{5.42}{m/s}$, $p = \unit{83667}{Pa}$
	\end{antwoord}
%%%%%%%%%%%%%%%%%%%%%%%%%%%%%%%%%%%%%%%%%%%%%%%%%%%%%%%%%%%%%%%%%%%%%%%%%%%%%%%%%%%%%%%%%
	\begin{toepassing*}
		\label{waterkraan}
Uit een kraan stroomt water verticaal naar beneden met een debiet van 5 l/min. Tegen de kraan heeft de waterstraal een diameter van 18mm. 

Bepaal de diameter van de waterstraal op een afstand van 0.2m onder de kraan.
		\begin{center}
			\includesvg{fig/behoudsvergelijkingen/waterkraan}
		\end{center}
	\end{toepassing*}
	\begin{antwoord}
		$d = \unit{5}{mm}$
	\end{antwoord}
%%%%%%%%%%%%%%%%%%%%%%%%%%%%%%%%%%%%%%%%%%%%%%%%%%%%%%%%%%%%%%%%%%%%%%%%%%%%%%%%%%%%%%%%%
	\begin{toepassing}
		\label{drukhoogte}
Een leidingstelsel is opgebouwd zoals weergegeven in onderstaande figuur. Er zijn twee verticale meetbuizen aangebracht om de druk in de leidingen te kunnen meten. In de leidingen stroomt water dat als niet viskeus beschouwd mag worden.
		
Bepaal de hoogte van de vloeistof in de twee meetbuizen t.o.v. het vloeistof niveau in de tank.
		\begin{center}
			\includesvg{fig/behoudsvergelijkingen/drukhoogte}
		\end{center}
	\end{toepassing}
	\begin{antwoord}
		$h_1 = \unit{400}{mm}$, $h_2 = \unit{25}{mm}$
	\end{antwoord}
%%%%%%%%%%%%%%%%%%%%%%%%%%%%%%%%%%%%%%%%%%%%%%%%%%%%%%%%%%%%%%%%%%%%%%%%%%%%%%%%%%%%%%%%%
	\begin{toepassing}
		\label{dynamische_druk}
Een een Pitot-buis en een manometer aangesloten op een leiding waarin lucht stroomt ($\rho=\unit{1.22}{kg/m^3}$). Beide buizen zijn aangesloten op een U-buis manometer waarin een olie met een dichtheid van \unit{800}{kg/m^3} zit. De hoogteverschillen in de U-buizen zijn 0.2m en 0.6m zoals aangegeven op de figuur.

Bepaal de statische en de dynamische druk in de buis en de snelheid in de buis.
		\begin{center}
			\includesvg{fig/behoudsvergelijkingen/dynamische_druk}
		\end{center}
	\end{toepassing}
	\begin{antwoord}
		$p_{\text{statisch}} = \unit{4709}{Pa}$, $p_{\text{dynamisch}} = \unit{1567}{Pa}$, $v = \unit{50.7}{m/s}$
	\end{antwoord}
%%%%%%%%%%%%%%%%%%%%%%%%%%%%%%%%%%%%%%%%%%%%%%%%%%%%%%%%%%%%%%%%%%%%%%%%%%%%%%%%%%%%%%%%%
	\begin{toepassing*}
		\label{turbine}
Een hydraulische turbine onttrekt 600MW arbeid aan een waterdebiet van \unit{600}{m^3/s}. De inlaat van de turbine heeft een diameter van 8m, de uitlaat een diameter van 10m. Aan de inlaat van de turbine is de druk 9 bar.

Bepaal de druk aan de uitlaat van de turbine.
		\begin{center}
			\includesvg{fig/behoudsvergelijkingen/turbine}
		\end{center}
	\end{toepassing*}
	\begin{antwoord}
		$p = \unit{}{bar}$
	\end{antwoord}
%%%%%%%%%%%%%%%%%%%%%%%%%%%%%%%%%%%%%%%%%%%%%%%%%%%%%%%%%%%%%%%%%%%%%%%%%%%%%%%%%%%%%%%%%
	\begin{toepassing}
		\label{vernauwing}
Een vernauwing in een buis heeft afmetingen zoals afgebeeld op de onderstaande figuur. Door de buis stroomt olie met een dichtheid van \unit{830}{kg/m^3}. De druk vlak voor de vernauwing is \unit{2}{bar}. De stroming mag stationair en zonder wrijving verondersteld worden.
		
Bepaal de kracht uitgeoefend op de vernauwing ten gevolge van de stroming.
		\begin{center}
			\includesvg{fig/behoudsvergelijkingen/vernauwing}
		\end{center}
	\end{toepassing}
	\begin{antwoord}
		$F = \unit{919}{N}$
	\end{antwoord}
%%%%%%%%%%%%%%%%%%%%%%%%%%%%%%%%%%%%%%%%%%%%%%%%%%%%%%%%%%%%%%%%%%%%%%%%%%%%%%%%%%%%%%%%%
	\begin{toepassing*}
		\label{brandslang}
Het uitlaatstuk van een brandslang vormt een vernauwing van de slang diameter (50mm) tot de uitlaat (16mm). De stroming hierin mag zonder wrijving verondersteld worden.

Bepaal de kracht die uitgeoefend wordt op het uitlaatstuk indien er een debiet van 180 l/min water door de slang stroomt en we de zwaartekracht niet beschouwen.
		\begin{center}
			%\includesvg{fig/behoudsvergelijkingen/brandslang}
		\end{center}
	\end{toepassing*}
	\begin{antwoord}
		$F_x = \unit{176}{N}$
	\end{antwoord}
%%%%%%%%%%%%%%%%%%%%%%%%%%%%%%%%%%%%%%%%%%%%%%%%%%%%%%%%%%%%%%%%%%%%%%%%%%%%%%%%%%%%%%%%%
	\begin{toepassing*}
		\label{45gradenbocht}
Een leiding met binnendiameter 300mm ligt in een horizontaal vlak. In de leiding stroomt water aan een debiet van \unit{0.25}{m^3/s}.    

Bepaal de grootte en de richting van de kracht die op het bochtstuk inwerkt indien de overdruk voor de bocht 0.4 bar bedraagt. De zwaartekracht mag buiten beschouwing gelaten worden.
		\begin{center}
			\includesvg{fig/behoudsvergelijkingen/45gradenbocht}
		\end{center}
	\end{toepassing*}
	\begin{antwoord}
		$F_x = \unit{1.1}{kN}$, $F_y = \unit{2.3}{kN}$
	\end{antwoord}
%%%%%%%%%%%%%%%%%%%%%%%%%%%%%%%%%%%%%%%%%%%%%%%%%%%%%%%%%%%%%%%%%%%%%%%%%%%%%%%%%%%%%%%%%
	\begin{toepassing*}
		\label{diffusiebocht}
Door een U-buis in een horizontaal vlak, met verlopende doorsnede, stroomt water met een intredesnelheid van \unit{4}{m/s}. De oppervlaktes van de doorsneden bij in en uittrede zijn gegeven in de figuur. De druk aan de intrede wordt gemeten en is \unit{150}{kPa}.
		
Bepaal de grootte en de richting van de reactiekracht van de buis op het water.
		\begin{center}
			\includesvg{fig/behoudsvergelijkingen/diffusiebocht}
		\end{center}
	\end{toepassing*}
	\begin{antwoord}
		$F = \unit{97.2}{kN}$
	\end{antwoord}
	
	\section*{Antwoorden}
	\begin{multicols}{2}
		\includecollection{antwoorden}
	\end{multicols}