% vaak gebruikte packages, nederlands
\usepackage[margin=2.5cm]{geometry}            % Marges instellen
\usepackage[dutch]{babel}               % Voor nederlandstalige hyphenatie (woordsplitsing)
\usepackage{amsmath,amsthm}             % Uitgebreide wiskundige mogelijkheden
\usepackage{url}                        % Om url's te verwerken
\usepackage{graphicx,subfigure}         % Om figuren te kunnen verwerken
\usepackage{color}
\usepackage{framed}
\usepackage{multicol}
\usepackage[small,bf,hang]{caption}     % Om de captions wat te verbeteren
\usepackage[utf8]{inputenc}             % Om niet ascii karakters rechtstreeks te kunnen typen
\usepackage{float}                      % Om nieuwe float environments aan te maken. Ook optie H!
\usepackage{flafter}                    % Opdat floats niet zouden voorsteken
\usepackage[section]{placeins}			% Om ervoor te zorgen dat floats binnen dezelfde section blijven
\usepackage[nottoc]{tocbibind}			% Bibliografie en inhoudsopgave in ToC; zie tocbibind.dvi
\usepackage{fancyhdr}                   % Voor fancy headers en footers
\usepackage{thmtools}                   % theorem tools
\usepackage{parskip}                    % Om paragrafen met een verticale spatie ipv horizontaal te laten beginnen
\usepackage[plainpages=false]{hyperref} % Om hyperlinks te hebben in het pdfdocument.



%%%%%%%%%%%%%%%%%%%%%%%%%%%%%%
% Algemene instellingen van het document.
%%%%%%%%%%%%%%%%%%%%%%%%%%%%%%
\renewcommand{\baselinestretch}{1.2} 	% De interlinie afstand wat vergroten.
\setcounter{MaxMatrixCols}{50}          % Max 20 kolommen in een matrix


%%%%%%%%%%%%%%%%%%%%%%%%%%%%%%
% Headers en footers
%%%%%%%%%%%%%%%%%%%%%%%%%%%%%%
\pagestyle{fancy}
\fancyhf{}
\renewcommand{\headrulewidth}{0pt}
\fancyhead[RO] {\rightmark}
\fancyhead[LE] {\leftmark}
\fancyfoot[RO,LE] {\thepage}

% page header and footer style in mainmatter aanpassen
\let\newmainmatter\mainmatter
\renewcommand{\mainmatter}{

	\pagestyle{fancy}
	\fancyhf{}
	\renewcommand{\headrulewidth}{0pt}
	\fancyhead[RO] {\rightmark}
	\fancyhead[LE] {\leftmark}
	\fancyfoot[RO,LE] {\thepage}
	\fancyfoot[C]{\tiny{Brecht Baeten}}

	\newmainmatter
}
\let\newappendix\appendix
\renewcommand{\appendix}{
	\fancyfoot{}
	\fancyfoot[RO,LE] {\thepage}
	\newappendix
}


%%%%%%%%%%%%%%%%%%%%%%%%%%%%%%
% Nieuwe omgevingen
%%%%%%%%%%%%%%%%%%%%%%%%%%%%%%
\definecolor{shadecolor}{gray}{0.98}
\newcounter{voorbeeldcounter}[chapter]
\renewcommand{\thevoorbeeldcounter}{\thechapter.\arabic{voorbeeldcounter}}
\makeatletter
\newenvironment{voorbeeld}
{
\vspace{3mm}
\addtolength{\leftskip}{5mm}
\begin{shaded*}
\vspace{-3mm}
\refstepcounter{voorbeeldcounter}
\noindent
\textbf{Voorbeeld \thevoorbeeldcounter:\\}
%\vspace{-8mm}
%\begin{multicols}{2}
}
{
%\end{multicols}
\end{shaded*}
\addtolength{\leftskip}{-5mm}
}
\makeatother  
    
%%%%%%%%%%%%%%%%%%%%%%%%%%%%%%
% .svg commando's
%%%%%%%%%%%%%%%%%%%%%%%%%%%%%%
% nieuw commando om svg files dynamisch te updaten
\newcommand{\executeiffilenewer}[3]{%
\ifnum\pdfstrcmp{\pdffilemoddate{#1}}%
{\pdffilemoddate{#2}}>0%
{\immediate\write18{#3}}\fi%
}
% nieuw commando om. svg figuren in te voegen
% Gebruik: \includesvg{path/filename.svg}
\newcommand{\includesvg}[2][0]{%
\executeiffilenewer{#2.svg}{#2.pdf}%
{inkscape -z -C --file=#2.svg %
--export-pdf=#2.pdf --export-latex}%
\ifx#10
	\let\svgwidth\undefined
\else
	\def\svgwidth{#1}
\fi%
\input{#2.pdf_tex}%
\ifx \svgwidth\undefined
\else
	\let\svgwidth\undefined
\fi%
}

% nieuw commando om .fig figuren in te voegen
\newcommand{\includefig}[2][0]{%
\ifx#10
	\let\figwidth\undefined
\else
	\def\figwidth{#1}
\fi%
\input{#2.pdf_tex}%
\ifx \figwidth\undefined
\else
	\let\figwidth\undefined
\fi%
}
