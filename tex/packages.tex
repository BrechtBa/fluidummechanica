%%%%%%%%%%%%%%%%%%%%%%%%%%%%%%
% Packages
%%%%%%%%%%%%%%%%%%%%%%%%%%%%%%
\usepackage{a4wide}                     % Iets meer tekst op een bladzijde
\usepackage[dutch]{babel}               % Voor nederlandstalige hyphenatie (woordsplitsing)
\usepackage{amsmath,amsthm}             % Uitgebreide wiskundige mogelijkheden
\usepackage{url}                        % Om url's te verwerken
\usepackage{graphicx}         			% Om figuren te kunnen verwerken
\usepackage[utf8]{inputenc}             % Om niet ascii karakters rechtstreeks te kunnen typen
\usepackage{parskip}					
\usepackage{thmtools}                   % theorem tools
\usepackage{collect}
\usepackage{multicol}

\usepackage[plainpages=false]{hyperref}    % Om hyperlinks te hebben in het pdfdocument.



%%%%%%%%%%%%%%%%%%%%%%%%%%%%%%
% Algemene instellingen van het document.
%%%%%%%%%%%%%%%%%%%%%%%%%%%%%%
%\setlength{\parindent}{0cm}             % Inspringen van eerste lijn van paragrafen is niet gewenst.
%\setlength{\parskip}{0.5cm}
%\renewcommand{\baselinestretch}{1.2} 	% De interlinie afstand wat vergroten.

% commando's voor Toepassing hoofdstukken in mainmatter
\addto\captionsdutch{\renewcommand\chaptername{Sessie}}



%%%%%%%%%%%%%%%%%%%%%%%%%%%%%%
% Environments
%%%%%%%%%%%%%%%%%%%%%%%%%%%%%%

% Een enviroment  voor toepassingen
\newlength{\currentparskip}
\newcounter{toepassingcounter}[chapter]
\renewcommand{\thetoepassingcounter}{\thechapter.\arabic{toepassingcounter}}
\newenvironment{toepassing}[1][]{
	\refstepcounter{toepassingcounter}
	\noindent
	\setlength{\currentparskip}{\parskip}
	\begin{minipage}{\textwidth}
	\setlength{\parskip}{\currentparskip}
	\textbf{#1Toepassing \thetoepassingcounter :}
}
{
	\vspace{0.5cm}
	\end{minipage}
}

% Een enviroment voor antwoorden
\definecollection{antwoorden}
\makeatletter
\newenvironment{antwoord}[1]{
	\@nameuse{collect}{antwoorden}{
		\noindent\vspace{0mm}
    	\textbf{#1:}
    }{}
}
{	
	\@nameuse{endcollect}	
}
\makeatother


% Een enviroment voor voorbeelden
\definecollection{voorbeelden}
\makeatletter
\newenvironment{voorbeeld}[1]{
	\@nameuse{collect}{voorbeelden}{
    	\noindent\vspace{0mm}
    	\textbf{#1:}
    }{}
}
{
	\@nameuse{endcollect}
}
\makeatother


%%%%%%%%%%%%%%%%%%%%%%%%%%%%%%
% Nieuwe commandos
%%%%%%%%%%%%%%%%%%%%%%%%%%%%%%

% De differentiaal operator
\newcommand{\diff}{\ensuremath{\mathrm{d}}}
\newcommand{\subsdiff}{\ensuremath{\mathrm{D}}}
\newcommand{\vardiff}{\ensuremath{\mathrm{\delta}}}

% Super en subscript
\newcommand{\supsc}[1]{\ensuremath{^{\text{#1}}}}   % Superscript in tekst
\newcommand{\subsc}[1]{\ensuremath{_{\text{#1}}}}   % Subscript in tekst

% Vectoren en matrices
\newcommand{\vt}[1]{\ensuremath{\boldsymbol{#1}}} % vector in juiste lettertype
\newcommand{\mx}[1]{\ensuremath{\mathsf{#1}}}	  % matrix in juiste lettertype

% Nieuw commando om iets te benadrukken en tegelijkertijd in de index te steken.
\newcommand{\begrip}[1]{\index{#1}\textbf{#1}\xspace}

% Graden celcius
\newcommand{\degC}{\ensuremath{^\circ \mathrm{C}}}
% graden
\renewcommand{\deg}{\ensuremath{^\circ}}

% unit
\newcommand{\unit}[1]{\ensuremath{\mathrm {#1}}}