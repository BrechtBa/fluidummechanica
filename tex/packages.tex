\usepackage{amsmath,amsthm}             % Uitgebreide wiskundige mogelijkheden
\usepackage{xcolor}						% Om kleuren te gebruiken

%%%%%%%%%%%%%%%%%%%%%%%%%%%%%%%%%%%%%%%%%%%%%%%%%%%%%%%%%%%%
% Nieuwe commandos
%%%%%%%%%%%%%%%%%%%%%%%%%%%%%%%%%%%%%%%%%%%%%%%%%%%%%%%%%%%%

% De differentiaal operator
\newcommand{\diff}{\ensuremath{\mathrm{d}}}
\newcommand{\subsdiff}{\ensuremath{\mathrm{D}}}
\newcommand{\vardiff}{\ensuremath{\mathrm{\delta}}}

% Super en subscript
\newcommand{\supsc}[1]{\ensuremath{^{\text{#1}}}}   % Superscript in tekst
\newcommand{\subsc}[1]{\ensuremath{_{\text{#1}}}}   % Subscript in tekst

% Vectoren en matrices
\newcommand{\vt}[1]{\ensuremath{\boldsymbol{#1}}} % vector in juiste lettertype
\newcommand{\mx}[1]{\ensuremath{\mathsf{#1}}}	  % matrix in juiste lettertype

% Nieuw commando om iets te benadrukken en tegelijkertijd in de index te steken.
\newcommand{\begrip}[1]{\index{#1}\textbf{#1}\xspace}

% Graden celcius
\newcommand{\degC}{\ensuremath{^\circ \mathrm{C}}}
% graden
\renewcommand{\deg}{\ensuremath{^\circ}}

% unit
\newcommand{\unit}[1]{\ensuremath{\mathrm {#1}}}


% underlinered
\newcommand{\underlinered}[1]{\color{red}\underline{{\color{black}#1}}\color{black}}