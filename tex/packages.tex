%%%%%%%%%%%%%%%%%%%%%%%%%%%%%%
% Packages
%%%%%%%%%%%%%%%%%%%%%%%%%%%%%%
\usepackage{a4wide}                     % Iets meer tekst op een bladzijde
\usepackage[dutch]{babel}               % Voor nederlandstalige hyphenatie (woordsplitsing)
\usepackage{amsmath,amsthm}             % Uitgebreide wiskundige mogelijkheden
\usepackage{amssymb}                    % Voor speciale symbolen zoals de verzameling Z, R...
\usepackage{url}                        % Om url's te verwerken
\usepackage{graphicx,subfigure}         % Om figuren te kunnen verwerken
\usepackage{color}                      % nodig voor svg's
\usepackage[utf8]{inputenc}             % Om niet ascii karakters rechtstreeks te kunnen typen
\usepackage{float}                      % Om nieuwe float environments aan te maken. Ook optie H!
\usepackage{flafter}                    % Opdat floats niet zouden voorsteken
\usepackage[section]{placeins}			% \FloatBarrier vormt een barriere voor floats
\usepackage{listings}                   % Voor het weergeven van letterlijke text en codelistings
\usepackage[numbers]{natbib}            % Voor juiste citatie stijl
\usepackage[nottoc]{tocbibind}			% Bibliografie en inhoudsopgave in ToC; zie tocbibind.dvi
\usepackage{eurosym}                    % om het euro symbool te krijgen
\usepackage{textcomp}                   % Voor onder andere graden celsius
\usepackage{fancyhdr}                   % Voor fancy headers en footers
\usepackage[Gray,squaren,thinqspace,thinspace]{SIunits} % Om elegant eenheden te zetten
\usepackage[version=3]{mhchem}          % Voor elegante scheikundige formules
\usepackage{emptypage}					% Om de lege pagina's voor een hoofdstuk mooi te maken
\usepackage{thmtools}                   % theorem tools
\usepackage{collect}
\usepackage{multicol}

\usepackage[plainpages=false]{hyperref}    % Om hyperlinks te hebben in het pdfdocument.




%%%%%%%%%%%%%%%%%%%%%%%%%%%%%%
% Algemene instellingen van het document.
%%%%%%%%%%%%%%%%%%%%%%%%%%%%%%

\setlength{\parindent}{0cm}             % Inspringen van eerste lijn van paragrafen is niet gewenst.
\setlength{\parskip}{0cm}
\renewcommand{\baselinestretch}{1.2} 	% De interlinie afstand wat vergroten.

\setcounter{MaxMatrixCols}{50}          % Max 20 kolommen in een matrix


% Vandaar dat we expliciet aangeven wanneer we wensen dat een nieuwe paragraaf begint:
% \par zorgt ervoor dat er een nieuwe paragraaf begint en
% \vspace zorgt voor verticale ruimte.
\newcommand{\npar}{\par \vspace{2.3ex plus 0.3ex minus 0.3ex}}

% commando's voor Toepassing hoofdstukken in mainmatter
\addto\captionsdutch{\renewcommand\chaptername{Sessie}}



%\newcommand{\includefigure}[1]{\vspace{0.5cm} \centering \includegraphics{#1}}


%%%%%%%%%%%%%%%%%%%%%%%%%%%%%%
% Environments
%%%%%%%%%%%%%%%%%%%%%%%%%%%%%%

% Een enviroment  voor toepassingen
\newcounter{toepassingcounter}[chapter]

\renewcommand{\thetoepassingcounter}{\thechapter.\arabic{toepassingcounter}}
\newenvironment{toepassing}{
	\refstepcounter{toepassingcounter}
	\noindent
	\begin{minipage}{\textwidth}
	\textbf{Toepassing \thetoepassingcounter :}
}
{
	\vspace{0.5cm}
	\end{minipage}
}


% Een enviroment  voor antwoorden
\definecollection{antwoorden}
\newcounter{antwoordcounter}[chapter]

\makeatletter
\newenvironment{antwoord}{
	\@nameuse{collect}{antwoorden}{
    	\refstepcounter{antwoordcounter}
    	\noindent\vspace{1mm}
    	\textbf{\arabic{chapter}.\arabic{antwoordcounter}:}
    }{}
}
{
	\@nameuse{endcollect}
}
\makeatother




%%%%%%%%%%%%%%%%%%%%%%%%%%%%%%
% Nieuwe commandos
%%%%%%%%%%%%%%%%%%%%%%%%%%%%%%

% De differentiaal operator
\newcommand{\diff}{\ensuremath{\mathrm{d}}}
\newcommand{\subsdiff}{\ensuremath{\mathrm{D}}}
\newcommand{\vardiff}{\ensuremath{\mathrm{\delta}}}

% Super en subscript
\newcommand{\supsc}[1]{\ensuremath{^{\text{#1}}}}   % Superscript in tekst
\newcommand{\subsc}[1]{\ensuremath{_{\text{#1}}}}   % Subscript in tekst

% Vectoren en matrices
\newcommand{\vt}[1]{\ensuremath{\boldsymbol{#1}}} % vector in juiste lettertype
\newcommand{\mx}[1]{\ensuremath{\mathsf{#1}}}	  % matrix in juiste lettertype

% Nieuw commando om iets te benadrukken en tegelijkertijd in de index te steken.
\newcommand{\begrip}[1]{\index{#1}\textbf{#1}\xspace}

% Graden celcius
\newcommand{\degC}{\ensuremath{^\circ \mathrm{C}}}


% nieuw commando om svg files dynamisch te updaten
\newcommand{\executeiffilenewer}[3]{%
\ifnum\pdfstrcmp{\pdffilemoddate{#1}}%
{\pdffilemoddate{#2}}>0%
{\immediate\write18{#3}}\fi%
}
% nieuw commando om. svg figuren in te voegen
% Gebruik: \includesvg{path/filename.svg}
\newcommand{\includesvg}[2][0]{%
\executeiffilenewer{#2.svg}{#2.pdf}%
{inkscape -z -C --file=#2.svg %
--export-pdf=#2.pdf --export-latex}%
\ifx#10
	\let\svgwidth\undefined
\else
	\def\svgwidth{#1}
\fi%
\input{#2.pdf_tex}%
\ifx \svgwidth\undefined
\else
	\let\svgwidth\undefined
\fi%
}

% nieuw commando om .fig figuren in te voegen
\newcommand{\includefig}[2][0]{%
\ifx#10
	\let\figwidth\undefined
\else
	\def\figwidth{#1}
\fi%
\input{#2.pdf_tex}%
\ifx \figwidth\undefined
\else
	\let\figwidth\undefined
\fi%
}