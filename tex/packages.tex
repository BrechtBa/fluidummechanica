%%%%%%%%%%%%%%%%%%%%%%%%%%%%%%
% Packages
%%%%%%%%%%%%%%%%%%%%%%%%%%%%%%
\usepackage[margin=2.5cm]{geometry}     % Marges instellen
\usepackage[dutch]{babel}               % Voor nederlandstalige hyphenatie (woordsplitsing)
\usepackage{amsmath,amsthm}             % Uitgebreide wiskundige mogelijkheden
\usepackage{url}                        % Om url's te verwerken
\usepackage{graphicx}         			% Om figuren te kunnen verwerken
\usepackage[utf8]{inputenc}             % Om niet ascii karakters rechtstreeks te kunnen typen
\usepackage{parskip}					% Om pafagrafen niet te laten inspringen
\usepackage{collect}					% Om antwoorden achteraan te verzamelen
\usepackage{multicol}					% Om secties met meerdere kolommen te maken
\usepackage{fancyhdr}                   % Voor fancy headers en footers

\usepackage[plainpages=false]{hyperref}    % Om hyperlinks te hebben in het pdfdocument.


%%%%%%%%%%%%%%%%%%%%%%%%%%%%%%
% Algemene instellingen van het document.
%%%%%%%%%%%%%%%%%%%%%%%%%%%%%%
% commando's voor Toepassing hoofdstukken in mainmatter
\addto\captionsdutch{\renewcommand\chaptername{Sessie}}


%%%%%%%%%%%%%%%%%%%%%%%%%%%%%%
% Headers en footers
%%%%%%%%%%%%%%%%%%%%%%%%%%%%%%
\pagestyle{fancy}
\fancyhf{}
\renewcommand{\headrulewidth}{0pt}
\fancyhead[RO] {\rightmark}
\fancyhead[LE] {\leftmark}
\fancyfoot[RO,LE] {\thepage}

% no dot after chapter number
\renewcommand{\chaptermark}[1]{
	\markboth{\MakeUppercase{ \chaptername\ \thechapter\quad #1}}{}
}

% page header and footer style in mainmatter aanpassen
\let\newmainmatter\mainmatter
\renewcommand{\mainmatter}{

	\pagestyle{fancy}
	\fancyhf{}
	\renewcommand{\headrulewidth}{0pt}
	\fancyhead[RO] {\rightmark}
	\fancyhead[LE] {\leftmark}
	\fancyfoot[RO,LE] {\thepage}
	\fancyfoot[C]{\tiny{Brecht Baeten}}

	\newmainmatter
}
\let\newappendix\appendix
\renewcommand{\appendix}{
	\fancyfoot{}
	\fancyfoot[RO,LE] {\thepage}
	\newappendix
}

%%%%%%%%%%%%%%%%%%%%%%%%%%%%%%
% Environments
%%%%%%%%%%%%%%%%%%%%%%%%%%%%%%
\raggedbottom

% Een enviroment  voor toepassingen
\newlength{\currentparskip}
\newcounter{toepassingcounter}[chapter]
\renewcommand{\thetoepassingcounter}{\thechapter.\arabic{toepassingcounter}}
\newenvironment{toepassing}[1][]{
	\refstepcounter{toepassingcounter}
	\noindent
	\setlength{\currentparskip}{\parskip}
	\begin{minipage}{\textwidth}
	\setlength{\parskip}{\currentparskip}
	\textbf{#1Toepassing \thetoepassingcounter :}
}
{
	\vspace{0.5cm}
	\end{minipage}
}

% Een enviroment voor antwoorden
\definecollection{antwoorden}
\makeatletter
\newenvironment{antwoord}[1]{
	\@nameuse{collect}{antwoorden}{
		\noindent\vspace{0mm}
    	\textbf{#1:}
    }{}
}
{	
	\@nameuse{endcollect}	
}
\makeatother


% Een enviroment voor voorbeelden
\definecollection{voorbeelden}
\makeatletter
\newenvironment{voorbeeld}[1]{
	\@nameuse{collect}{voorbeelden}{
    	\noindent\vspace{0mm}
    	\textbf{#1:}
    }{}
}
{
	\@nameuse{endcollect}
}
\makeatother