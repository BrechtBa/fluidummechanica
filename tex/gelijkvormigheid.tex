\FloatBarrier
\chapter{Gelijkvormigheid en uitwendige stroming}
\label{sec:Gelijkvormigheid}
	\begin{toepassing*}
		\label{weerballon}
		De ontwerpers van een weerballon willen achterhalen wat de luchtweerstand van de ballon is bij de maximaal verwachte relatieve windsnelheid van 5 m/s bij standaard atmosfeercondities ($\rho = \unit{1.225}{kg/m^3}$, $\mu = \unit{1.83\times 10^{-5}}{Pa \cdot s}$ ).
		Hiervoor wil men een 1:20 schaalmodel testen in water bij 20\textcelsius\ ($\rho = \unit{998}{kg/m^3}$, $\mu = \unit{1\times 10^{-3}}{Pa \cdot s}$ ). 
		
		Bij welke watersnelheid moet men de tests uitvoeren opdat de data een voorspelling kunnen leveren van het gedrag van de balon in lucht? 
		
		Als er een dragkracht van \unit{2}{kN} wordt opgemeten bij deze snelheid, wat is dan de verwachte weerstandskracht bij het prototype?
	\end{toepassing*}
	\begin{antwoord}
		$v = \unit{6.71}{m/s}$, $F = \unit{0.55}{kN}$
	\end{antwoord}
%%%%%%%%%%%%%%%%%%%%%%%%%%%%%%%%%%%%%%%%%%%%%%%%%%%%%%%%%%%%%%%%%%%%%%%%%%%%%%%%%%%%%%%%%%%%
	\begin{toepassing*}
		\label{insect}
		Het aerodynamisch gedrag van een vliegend insect wordt bestudeerd in een windtunnel met een 10:1 schaalmodel. In werkelijkheid klappert het insect 50 maal per seconde met de vleugels en vliegt het met een snelheid van \unit{1.25}{m/s} ten opzichte van de lucht. 
		
		Kan volledige gelijkvormigheid behaald worden?
		
		Bepaal de snelheid in de windtunnel en de klapfrequentie van het schaalmodel opdat de uitgevoerde metingen representatief zouden zijn voor de werkelijkheid.
	\end{toepassing*}
	\begin{antwoord}
		$v = \unit{0.125}{m/s}$, $f = \unit{0.5}{Hz}$
	\end{antwoord}
%%%%%%%%%%%%%%%%%%%%%%%%%%%%%%%%%%%%%%%%%%%%%%%%%%%%%%%%%%%%%%%%%%%%%%%%%%%%%%%%%%%%%%%%%%%%
	\begin{toepassing*}	
		\label{onderzeeer}	
		Een onderzeeër moet tegen een snelheid van 10 knopen (1 knoop = 0,5144 m/s) onder zeewater ($\rho = \unit{1028}{kg/m^3}$, $\nu = \unit{1.83}{mm^2/s}$) kunnen varen. Er wordt een experiment gedaan op schaal 1:20 in zoetwater ($\rho = \unit{998}{kg/m^3}$, $\nu = \unit{1.00}{mm^2/s}$).
		
		Bepaal de snelheid waarop het model getest moet worden.
		
		Bepaal het vermogen dat de onderzeeër voor voortstuwing nodig heeft als op het model een kracht van \unit{200}{kN} wordt gemeten.
	\end{toepassing*}
	\begin{antwoord}
		$v = \unit{56.2}{m/s}$, $P = \unit{3549}{kW}$
	\end{antwoord}
%%%%%%%%%%%%%%%%%%%%%%%%%%%%%%%%%%%%%%%%%%%%%%%%%%%%%%%%%%%%%%%%%%%%%%%%%%%%%%%%%%%%%%%%%%%%
\begin{toepassing*}
	\label{schip}
Een schip van \unit{152.4}{m} lang vaart met een snelheid van 15 knopen (1 knoop = 0,5144 m/s). Een model op schaal 1:25 wordt getest in hetzelfde water.

Bepaal de snelheid van het model voor :
	\begin{enumerate}
		\item Dynamische gelijkvormigheid voor oppervlaktegolven.
		\item Dynamische gelijkvormigheid voor wrijving.
	\end{enumerate}
\end{toepassing*}
\begin{antwoord}
	$v_{\text{oppervlaktegolven}} = \unit{1.54}{m/s}$, $v_{\text{wrijving}} = \unit{192.9}{m/s}$
\end{antwoord}
