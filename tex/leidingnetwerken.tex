\chapter{Leidingnetwerken}
\label{sec:Leidingnetwerken}
\begin{toepassing}
	\label{parallelle leidingen}
Twee open reservoirs zijn met elkaar verbonden door 3 afzonderlijke leidingen met diameters $d$, $2d$, $3d$. De leidingen hebben allen dezelfde lengte.
		
	Veronderstel dat de wrijvingsfactoren gelijk zijn, wat zal dan het volumedebiet doorheen de twee grootste leidingen zijn als het debiet door de kleinste leiding gelijk is aan \unit{0.03}{m^3/s}? 
\end{toepassing}
\begin{antwoord}
	$\dot{V}_2 = \unit{0.17}{m^3/s}$, $\dot{V}_3 = \unit{0.47}{m^3/s}$
\end{antwoord}
%%%%%%%%%%%%%%%%%%%%%%%%%%%%%%%%%%%%%%%%%%%%%%%%%%%%%%%%%%%%%%%%%%%%%%%%%%%%%%%%%%%%%%
\begin{toepassing*}
	\label{gesplitste leiding}
Twee open reservoirs, waarvan het verschil tussen de twee vloeistofniveaus 6 meter is, zijn door een leiding met elkaar verbonden. De leiding ($d=\unit{600}{m}$) vertrekt in het reservoir met het hoogste waterniveau. Na \unit{3000}{m} splits te leiding in 2 leidingen ($d=\unit{300}{m}$) van elk \unit{3000}{m} lang. Deze leidingen komen beide aan in het tweede reservoir. 
		
	Als de wrijvingsfactor gelijk is aan \unit{0.04}{} bepaal dan het totale volumedebiet. 
\end{toepassing*}
\begin{antwoord}
		$\dot{V} = \unit{0.723}{m^3/s}$
\end{antwoord}
%%%%%%%%%%%%%%%%%%%%%%%%%%%%%%%%%%%%%%%%%%%%%%%%%%%%%%%%%%%%%%%%%%%%%%%%%%%%%%%%%%%%%%
\begin{toepassing*}
	\label{3_reservoirs}
Drie open reservoir zijn verbonden zoals op onderstaande afbeelding. De wrijvingsfactor van de drie leidingen is 0.032. In en uitstroom verliezen worden verwaarloosd.
		
	Bepaal het debiet in alle drie de leidingen en de overdruk in het knooppunt.
		
	\begin{center}
		\includesvg{fig/leidingnetwerken/3_reservoirs}
	\end{center}
\end{toepassing*}
\begin{antwoord}
		$\dot{V}_1 = \unit{764}{l/min}$, $\dot{V}_2 = \unit{128}{l/min}$, $\dot{V}_3 = \unit{893}{l/min}$, $p=\unit{-31.6}{kPa}$
\end{antwoord}
%%%%%%%%%%%%%%%%%%%%%%%%%%%%%%%%%%%%%%%%%%%%%%%%%%%%%%%%%%%%%%%%%%%%%%%%%%%%%%%%%%%%%%
\begin{toepassing*}
	\label{hoogteverschil}
Een hoofdleiding met een diameter van \unit{675}{mm} loopt horizontaal over \unit{1500}{m} en splitst dan in 2 dezelfde leidingen $d=\unit{450}{mm}$ en $l=\unit{3000}{m}$. Beide leidingen zijn open op het einde en komen op de zelfde eindhoogte uit, maar één van de twee leidingen heeft ook openingen in de zijwand. Hierdoor gaat de helft van het water afgevoerd worden over de totale lengte van de leiding.
	Als de wrijvingsfactor 0.024 is voor alle leidingen en het debiet is \unit{0.28}{m^3/s} in de hoofdleiding, bereken dan het hoogteverschil tussen het vloeistof oppervlak en de plaats waar de vloeistof in de atmosfeer komt. Men mag de vertragingsverliezen verwaarlozen.
\end{toepassing*}
\begin{antwoord}
	$\Delta h = $
\end{antwoord}
%%%%%%%%%%%%%%%%%%%%%%%%%%%%%%%%%%%%%%%%%%%%%%%%%%%%%%%%%%%%%%%%%%%%%%%%%%%%%%%%%%%%%%
\begin{toepassing}
	\label{geperforeerde leiding}
Een open reservoir voedt een leiding van \unit{300}{m} lang, diameter \unit{200}{mm}. Deze leiding splitst in 2 gelijke leidingen \unit{150}{mm} in diameter en \unit{150}{m} lang. Beide leidingen zijn open op het einde en bevinden zich beide op een hoogte van \unit{15}{m} onder het vloeistofoppervlak van het reservoir. E\'en leiding heeft een uniforme afvoer langs de zijwand en verliest over de volledige lengte de helft van de hoeveelheid water door die leiding. De wrijvingsfactor is \unit{0.024}{} voor beide leidingen. Alle lokale verliezen mogen verwaarloosd worden.
		
	Bereken het volumedebiet van de twee leidingen afzonderlijk. 
\end{toepassing}
\begin{antwoord}
	$\dot{V}_1 = \unit{0.034}{m^3/s}$, $\dot{V}_2 = \unit{0.045}{m^3/s}$
\end{antwoord}
%%%%%%%%%%%%%%%%%%%%%%%%%%%%%%%%%%%%%%%%%%%%%%%%%%%%%%%%%%%%%%%%%%%%%%%%%%%%%%%%%%%%%%
\begin{toepassing}
	\label{3uitlaten}
Een ventilatieleiding met diameter 300mm en ruwheid 1mm heeft 3 regelbare uitlaten op onderling gelijke afstand van 12m. De ventielen moeten zo geregeld worden dat door elke uitlaat hetzelfde debiet stroomt. $\rho = \unit{1.22}{kg/m^3}$, $\nu = \unit{15}{mm^2/s}$.
	
	Bepaal het ladingsverlies dat moet ingesteld worden aan de eerste twee uitlaten indien er vlak voor de eerste uitlaat een statische overdruk heerst van 50Pa en de laatste uitlaat een ladingsverlies van 5.00m veroorzaakt.
	\begin{center}
		\includesvg{fig/leidingnetwerken/3uitlaten}
	\end{center}
\end{toepassing}
\begin{antwoord}
	$h_{l,1} = \unit{6.98}{m}$, $h_{l,2} = \unit{5.41}{m}$
\end{antwoord}
%%%%%%%%%%%%%%%%%%%%%%%%%%%%%%%%%%%%%%%%%%%%%%%%%%%%%%%%%%%%%%%%%%%%%%%%%%%%%%%%%%%%%%
\begin{toepassing*}
	\label{grondwarmtewisselaar}
Een grondwarmtewisselaar bestaat uit 2 paralelle boringen van elk 100m diepte. De twee boringen liggen 50m uit elkaar. Het volledige systeem is opgebouwd uit leidingen met een binnendiameter van 37mm en een ruwheid van 0.015mm. Door de leidingen stroomt een anti-vries oplossing met een dichtheid van \unit{1130}{kg/m^3} en een viscositeit van \unit{1.463}{mPa s}.
	
	Bepaal de het debiet door de twee boringen indien in de hoofdleiding een debiet van \unit{20}{l/min} stroomt.
	\begin{center}
		\includesvg{fig/leidingnetwerken/grondwarmtewisselaar}
	\end{center}
\end{toepassing*}
\begin{antwoord}
	$\dot{V}_{1} = \unit{11.6}{l/min}$, $\dot{V}_{2} = \unit{8.4}{l/min}$
\end{antwoord}
%%%%%%%%%%%%%%%%%%%%%%%%%%%%%%%%%%%%%%%%%%%%%%%%%%%%%%%%%%%%%%%%%%%%%%%%%%%%%%%%%%%%%%
	\section*{Antwoorden}
\begin{multicols}{2}
	\includecollection{antwoorden}
\end{multicols}