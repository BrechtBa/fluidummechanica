\FloatBarrier
\chapter{Inwendige stroming}
\label{sec:Inwendige stroming}

	\FloatBarrier
	\section{Inleiding}
	\label{sec:Inwendige stroming Inleiding}

Opbouw van een grenslaag aan de wanden van een buis.
	\FloatBarrier
	\section{Dimensie analyse van de drukval in een cilindrische leiding}
	\label{sec:Dimensie analyse van de drukval in een cilindrische leiding}
Het is interessant om voor het bepalen van de drukval voor specifieke situaties een dimensie analyse uit te voeren. Hiermee kunnen we onze resultaten controleren en in een toepasselijke vorm schrijven.
\npar
Uit experimenten, of fysiche inzicht kunnen we beredeneren dat de drukval in een cilindrische leiding afhankelijk zal zijn van de lengte van de leiding, de diameter, de gemiddelde snelheid van de stroming, de viscositeit en de dichtheid.
dit kunnen we schrijven als:
\begin{equation}
	\Delta p = \phi(L,D,v,\mu,\rho)
\end{equation}
Volgens het Buckingham-pi theorema kunnen we dus de dimensieloze drukval schrijven als functie van 2 dimensieloze parameters ($n=5$,$k=3$). Deze parameters kunnen we eenvoudig vinden als de relatieve lengte ($L/D$) en het Reynoldsgetal ($\rho v D/\mu = v D/\nu$). Om de druk dimensieloos te maken kunnen we deze delen door de snelheidsdruk ($\frac{1}{2} \rho v^2$):
\begin{equation}
	\frac{\Delta p}{\frac{1}{2}\rho v^2} = f(L/D,Re)
\end{equation}
Uit logische overwegingen moet hete drukverlies evenredig zijn met de lengte van de buis:
\begin{equation}
	\frac{\Delta p}{\frac{1}{2}\rho v^2} = \frac{L}{D} f(Re)
	\label{eqn:dimensie analyse drukval laminair}
\end{equation}
Het dimensieloze drukverschil zal dus gelijk zijn aan de relatieve lengte van de buis vermenigvuldigd met een nog onbekende functie van het Reynoldsgetal. Deze onbekende functie noemen we de wrijvingsfactor. We zullen nu trachten het drukverschil te berekenen en in deze vorm te schrijven.
	
	\FloatBarrier
	\section{Laminaire stroming in een cilindrische leiding}
	\label{sec:Laminaire stroming in een cilindrische leiding}
Beschouw een stationaire, laminaire, volledig ontwikkelde stroming in een cilindrische buis (Figuur \ref{fig:laminaire_stroming_in_buis}).

\begin{figure}
	\centering
	\includesvg{fig/inwendige_stroming/Laminaire_stroming_in_buis}
	\caption{Laminaire stroming in een cilyndrische buis}
	\label{fig:laminaire_stroming_in_buis}
\end{figure}
Om het snelheidsprofiel van de stroming uit te rekenen beschouwen we een cylindrisch controle volume en passen we hier de wet van behoud van impuls (\ref{eqn:controlevolume,behoud van impuls}) op toe in de stroomrichting. Aangezien de stroming stationair is zal de verandering van impuls binnen het controle volume $0$ zijn. Aangezien de stroming laminair is, is het controlevolume een deel van een stroombuis. De netto uitstroom van impuls uit het controlevolume is dus ook $0$. Vergelijking \ref{eqn:controlevolume,behoud van impuls} wordt dan:
\begin{equation}
	F_x = 0
\end{equation}
De inwerkende krachten bestaan in dit geval uit drukkrachten aan de linker en rechterzijde van het controlevolume en een schuifpanning aan de omtrek van het controlevolume. Dit geeft:
\begin{equation}
	\left. p \pi r^2\right|_{x} - \left. p \pi r^2\right|_{x+\Delta x} -  \tau 2 \pi r \Delta x = 0
\end{equation}
Na deling door $2\pi r \Delta x$ en de limiet te nemen voor $\Delta x$ gaande naar $0$ bekomen we dit:
\begin{equation}
	\frac{1}{2} \frac{\diff p}{\diff x} r= \tau
	\label{eqn:drukgradient}
\end{equation}
Indien we een Newtoniaanse vloeistof veronderstellen kunnen we de schuifspanning schrijven als:
\begin{equation}
	\tau = -\mu \frac{\diff v}{\diff r}
\end{equation}
Het min-teken wordt hier ingevoerd aangezien de afstand tot de wand afneemt met stijgende $r$. Dit resultaat invullen in (\ref{eqn:drukgradient}) geeft:
\begin{equation}
	\frac{1}{2} \frac{\diff p}{\diff x} r = -\mu \frac{\diff v}{\diff r}
\end{equation}
Of na omvorming:
\begin{equation}
	\frac{\diff v}{\diff r} = \frac{1}{2 \mu}\frac{\diff p}{\diff x} r
\end{equation}
Deze vergelijking kunnen we integreren over de straal. We bekomen dan een uitdrukking voor de snelheid.
\begin{equation}
	v = \frac{1}{4 \mu}\frac{\diff p}{\diff x} r^2 + C
\end{equation}
De integratie constante $C$ halen we uit de no-slip randvoorwaarde:
\begin{equation}
	\left.v\right|_{r=R} = \frac{1}{4 \mu}\frac{\diff p}{\diff x} R^2 + C  = 0
\end{equation}
Dus:
\begin{equation}
	C = - \frac{1}{4 \mu}\frac{\diff p}{\diff x} R^2
\end{equation}
De het snelheidsprofiel van een stationaire, laminaire, Newtoniaanse vloeistof in een cilyndrische buis wordt dan:
\begin{equation}
	v = - \frac{1}{4 \mu}\frac{\diff p}{\diff x} R^2 \left(1- \frac{r^2}{R^2}\right)
\end{equation}
We vinden een kwadratisch snelheidsprofiel terug met als maximale snelheid:
\begin{equation}
	v_{\text{max}} = - \frac{1}{4 \mu}\frac{\diff p}{\diff x} R^2
\end{equation}
Een stroming met dit soort snelheidsprofiel wordt poiseuille stroming genoemd naar de franse fysicus Jean Louis Marie Poiseuille en is weergegeven in Figuur \ref{fig:laminair_snelheidsprofiel}:
\begin{figure}
	\centering
	\includesvg{fig/inwendige_stroming/Laminair_snelheidsprofiel}
	\caption{Snelheidsprofiel bij laminaire stroming in een cilyndrische buis}
	\label{fig:laminair_snelheidsprofiel}
\end{figure}
\npar
Voor het maken van berekeningen is het interessant om de drukverandering te kunnen schrijven in functie van de gemiddelde snelheid in de buis. We kunnen de gemiddelde snelheid in de buis berekenen door het debiet te delen door de oppervlakte. Het debiet in de buis kan berekend worden als de integraal van de snelheid over de volledige oppervlakte:
\begin{equation}
	\dot{V} = 2 pi \int_0^R v_{\text{max}} \left(1- \frac{r^2}{R^2}\right) = v_{\text{max}} \frac{\pi R^2}{2}
\end{equation}
De gemiddelde snelheid wordt dan:
\begin{equation}
	v_{\text{gem}} = \frac{v_{\text{max}}}{2} = - \frac{1}{8 \mu}\frac{\diff p}{\diff x} R^2
\end{equation}
Voor ingenieurs is het handiger om met de diameter $D$ van de buis te werken in plaats van met de straal $R$. We kunnen het drukverlies over een horizontale leiding met lengte $L$ dan uitdrukken als:
\begin{equation}
	\Delta p = - 32 \mu \frac{v_{\text{gem}} L}{ D^2}
\end{equation}
Uit de dimensie analyse weten we dat de drukval afhankelijk moet zijn van $\frac{1}{2}\rho v^2$ aangezien deze samen de dimensieloze drukcoefficient vormen. Indien we het vorige resultaat herschrijven bekomen we inderdaad:
\begin{equation}
	\Delta p = 64 \frac{1}{2}\rho v^2 \frac{\mu}{v D} \frac{L}{D} = \frac{64}{Re} \frac{1}{2}\rho v^2 \frac{L}{D}
	\label{eqn:drukval bij laminaire stroming}
\end{equation}
We hebben nu een vergelijking om de drukval over een cylindrische leiding voor laminaire, niet-samendrukbare stroming te berekenen. Deze vergelijking staat bekend als de wet van Hagen-Poiseuille. Hierin is de drukcoefficient afhankelijk van twee dimensieloze combinaties, $64/Re$ en $L/D$, zoals voorspeld in sectie \ref{sec:Dimensie analyse van de drukval in een cilindrische leiding}. Voor laminaire stroming is de wrijvingsfactor $f$ dus gelijk aan $64/Re$.

	\FloatBarrier
	\section{Turbulente stroming}
	\label{sec:Turbulente stroming}

Wanneer de diameter van de buis of de snelheid in de buis groot genoeg is zal de stroming in de buis niet meer laminair zijn. Er zijn fluctuaties in de snelheid die ervoor zorgen dat het snelheidsprofiel niet meer stationair is. Dit gedrag werd experimenteel waargenomen door Osbourne Reynolds. Hij zag dat wanneer de dimensiloze combinatie $\frac{v D}{\nu}$ laag genoeg bleef, de stroming in een buis laminair was. De dimensieloze cominatie is dan ook naar hem vernoemd en kennen we als het Reynoldsgetal.
\npar
In cilindrische buizen blijft de stroming laminair tot een Reynoldsgetal van ongeveer 2300. Bij een Reynoldsgetal groter dan 10000 zal de stroming daarintegen meestal turbulent zijn. Bij Reynoldsgetallen tussen 2300 en 10000 is het zowel mogelijk dat de stroming laminair of turbulent is. Deze zone noemt met het overgangsgebied.
\npar
Wanneer we dit vergelijken met de waarnemingen die we gedaan hebben bij de vlakke wand in Sectie \ref{sec:Grenslagen}, zien we een gelijkaardig gedrag. In een Buis als er een axis-symetrische grenslaag ontstaan die in het midden van de buis samenkomt. Wanneer de diameter van de buis dus klein genoeg blijft (kleiner dan de critische grenslaag dikte) zal de stroming laminair blijven. Wanneer de diameter van de buis groter wordt wardt de stroming turbulent. Ook de manier waarop de weerstandskracht uitgeoefend wordt verloopt gelijkaardig. In het centrale turbulente gedeelte is er impulsoverdracht (of wrijving) door de onregelmatigheid van de snelheid (turbulente schuifspanning). In de laminaire sublaag dicht bij de wand is er wrijving door de viscositeit (laminaire schuifspanning). Er ontstaat dus een centraal vlakker snelheidsprofiel met een zeer plots overgang aan de wanden van de buis (Figuur \ref{fig:turbulent_snelheidsprofiel}). Deze grote snelheidsgradi\"ent brengt grote schuifspanningen en dus een grotere drukval met zich mee. 
\begin{figure}
	\centering
	\includesvg{fig/inwendige_stroming/Turbulent_snelheidsprofiel}
	\caption{Snelheidsprofiel bij turbulente stroming in een cilyndrische buis}
	\label{fig:turbulent_snelheidsprofiel}
\end{figure}
\npar
Bij metingen van het drukverlies bij turbulente stroming door een horizontale buis blijkt dat deze afhankelijk wordt van het type buis dat gebruikt werd. Meer bepaald de ruwheid van de buis zal een rol te spelen. We zullen dus (\ref{eqn:dimensie analyse drukval laminair}) moeten aanpassen. Dit kan eenvoudig door de relatieve ruiwheid te defini\"eren als de ruwheid gedeeld door de diameter. De dimensieloze relatie wordt dan:
\begin{equation}
	\frac{\Delta p}{\frac{1}{2}\rho v^2} = \frac{L}{D} f(Re,\varepsilon/D)
	\label{eqn:dimensie analyse drukval turbulent}
\end{equation}
De wrijvingsfactor wordt dus afhankelijk van het Reynoldsgetal en de relatieve ruwheid (De geruikte waarde van ruwheid komt overeen met de diameter van de ruwheidskorrels). De afhankelijk van de relatieve ruwheid kunnen we verklaren door naar de laminaire sublaag aan de wand van de buis te kijken (\ref{fig:Invloed_ruwheid}). Wanneer de wand zeer ruw is zullen de ruwheidspieken tot ver in de laminaire sublaag doordringen. De laminaire schuifspanning zal dus beinvloed worden door de ruwheid. Bij laminaire stroming strekt de laminaire laag zich uit over heel de buis. De ruwheid zal dus bijna geen invloed hebben op de totale weerstand.
\begin{figure}
	\centering
	\includesvg{fig/inwendige_stroming/Invloed_ruwheid}
	\caption{De invloed van de relatieve ruwheid op de laminaire sublaag}
	\label{fig:Invloed_ruwheid}
\end{figure}

\npar
Waarden van karakteristieke ruwheden voor verschillende materialen zijn weergegeven in Tabel \ref{tab:ruwheid van materialen}
\begin{table}
	\centering
	\begin{tabular}{cc}
		\hline
		Oppervlak & Ruwheid \\
		   & (mm) \\
		\hline
		Commercieel glad messing, lood, koper of kunststof & 0.0015 \\
		Staal en smeedijzer & 0.046 \\
		Gegalvanizeerd staal of ijzer & 0.152 \\
		Gietijzer & 0.259 \\
		\hline
	\end{tabular}
	\caption{Karakteristieke ruwheidswazarden voor verschillende materialen (uit ASHRAE Handbook of Fundamentals \cite{ASHRAE_Fundamentals})}
	\label{tab:ruwheid van materialen}
\end{table}
Het drukverlies wordt dus:
\begin{equation}
	\Delta p = f \frac{1}{2}\rho v^2 \frac{L}{D}
	\label{eqn:drukval bij turbulente stroming}
\end{equation}
\npar
Na vele experimenten door onder andere Johann Nikuradse (1933) werd de relatie tussen de wrijvingsfactor, het Reynoldsgetal en de relatieve ruwheid vastgesteld. Deze relatie kan benaderd worden door de formule van Colebrook. Deze heeft een relatieve naukeurigheid van 5\% voor gladde leidingen en 10\% voor ruwe leidingen.
\begin{equation}
	\frac{1}{\sqrt{f}} = -2 \log \left( \frac{\varepsilon/D}{3.72} + \frac{2.51}{Re \sqrt{f}} \right)
\end{equation}
Dit is een impliciete formule die niet in gesloten vorm naar $f$ kan worden opgelost. Ze kan wel iteratief worden opgelost.  Om dit probleem te verhelpen wordt de wrijvingsfactor in grafische vorm weergegeven in het Moody diagram (Appendix \ref{fig:Moody diagram}) genoemd naar Lewis F. Moody die het als eerste voorstelde in 1944.
\npar
Wanneer we het diagram bestuderen zien we dat bij grote waarden van het Reynolds getal de wrijvingsfactor niet meer afhankelijk is van het Reynoldsgetal maar enkel nog van de relatieve ruwheid. In dit geval wordt de stroming volledig ruw genoemd.

	\FloatBarrier
	\subsection{Stroming door niet cilindrische buizen}
Bij volledig ontwikkelde stroming door volledig gevulede niet cilindrische buizen of kanalen kunnen gelijkaardige formules als hierboven toegepast worden op voorwaarde dat de diameter $D$ van de buis wordt vervangen door de hydraulische diameter $D_h$ gedefinieerd als:
\begin{equation}
	D_h = \frac{4 A}{P}
\end{equation}
Met $A$ de doorstroom sectie en $P$ de bevochtigde omtrek van de sectie.
	


